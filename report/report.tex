\documentclass{report}

\usepackage{graphicx}
\usepackage{subfigure}

\usepackage{algorithm}
\usepackage{algpseudocode}
\renewcommand{\algorithmicforall}{\textbf{for each}}
\def\ForEach{\ForAll}

\title{Random Walks course project: Simulated Annealing Algorithm for Graph Coloring}

\author{
  R\'oger Berm\'udez Chac\'on\\EPFL
  \and
  Victor Kristof\\EPFL
  \and
  Merlin Nimier-David\\EPFL
}

\begin{document}
  \maketitle

  % -----------------------------------------------------------------------------------
  \section*{Problem statement}
  \paragraph{Proper q-colorings}
  TODO: Define the problem. Show how the problem is posed as an optimization (cost, etc).

  \paragraph{Input data}
  TODO: Describe construction of the random graph. Describe parameters. Show examples for different density parameters.

  % -----------------------------------------------------------------------------------
  \section*{Metropolis solution}
  \paragraph{TODO} Describe how the optimization can be solved using Metropolis. Explain why we're convinced it will converge if there is a proper coloring.

  % -----------------------------------------------------------------------------------
  \section*{Simulated annealing}
  \paragraph{TODO} Describe the intuition behind the ``temperature''. Define what's a schedule.

  % -----------------------------------------------------------------------------------
  \section*{Results}
  \paragraph{Implementation details}
  TODO: Efficient implementations for generating the graph, transitioning, etc. Give some number to describe algorithm's performance (e.g. average time / iteration).

  \paragraph{Experiments}
  TODO: Explain the different schedules that we tried. Give specific numbers for each parameter (min / max temperature used, starting point, etc). Show results plot for each. Pick the best one and give intuition as to why it performs better than the others.

\end{document}

